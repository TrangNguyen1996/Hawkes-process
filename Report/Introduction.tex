\chapter{Introduction}
\label{chapter:intro}
In reality, an earthquake can increase the geological tension in the region where it occurs. Selling a significant quantity of a stock could cause a trading flurry. These above problems are called the applications of self-exciting stochastic processes. This process is called Hawkes process which was proposed by Alan G. Hawkes \cite{hawkes}. \\
The Hawkes process is a counting process that models a sequence of 'arrivals' of some types over time, for example, earthquakes, trade orders. Each arrival excites the process, i.e., the chance of a next arrival is increased for some time period after the initial arrival.\\
%and it is based on a counting process in which the intensity function depends explicitly an all previously occurred events. 
The outline of this report is organized as follows:

	Chapter \ref{chapter:knowledge} reviews and introduces some basis knowledge. It will support us in the next chapters such as point process, counting process, conditional intensity function, inhomogeneous Poisson process.
	
	Chapter \ref{chapter:hawkes} approaches two models to simulate Hawkes process: the intensity-based Hawkes process and cluster-based Hawkes process.
	
	Chapter \ref{chapter:simulation} is devoted to some simulations by using Matlab\textsuperscript{\textregistered} code such as the inhomogeneous Poisson process and intensity-based Hawkes process and cluster-based Hawkes process.
	
	Chapter \ref{chapter:application} will introduce an application of Hawkes process in seismology.
	
	Chapter \ref{chapter:conclusion} summarizes the content of the report what has been done and limited of its.\\
%	summarizes the thesis and analysisnof the results. We also discuss how to apply the method to other applications.
	